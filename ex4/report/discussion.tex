\section*{General description}

Our implementation of the Simulated Annealing algorithm is a pretty straightforward translation of the pseudocode in the handout.
We have wrapped the search procedure in a class encapsulating some of the properties of an SA-problem for convenience.
The starting solution is the only parameter passed to the search function itself.


\section*{The Egg Carton Puzzle}

For an $M*N$ carton with limit $K$, the optimal solution will have $\text{NUM\_EGGS} = K * \min (M, N)$.
(The exception to this rule is the case of $K = \min(M, N)-1$.)
We use this to simplify the problem. The start should be a carton with $\text{NUM\_EGGS}$ already distributed.
A neighboring carton is generated by swapping one egg to an open slot, keeping the number of eggs constant.

With this known, our obj\_func does not need to penalize open slots in any way,
because we know the number of eggs is correct, it is just their placement that is to be considered.
The obj\_func therefore penalizes one point per egg over $K$ in each row, column and diagonal.
The sum of all penalties is then compared to the absolute worst score achievable (eggs in every slot),
resulting in a score between 0 (worst) and 1 (optimal solution).

Testing $\text{grid\_score}$ with SOLUTION552 gives a score of 1.

\begin{figure}[h!]
    \begin{verbatim}
    $ python3.4 eggses.py
    . . O O . | 2
    . . O O . | 2
    . . O . O | 2
    . O . O O | 3
    O . . . . | 1
    - - - - -
    1 1 3 3 2
    Score: 0.875

    . . O . O | 2
    O . O . . | 2
    O . . O . | 2
    . O . O . | 2
    . O . . O | 2
    - - - - -
    2 2 2 2 2
    Score: 1.0
    \end{verbatim}
\caption{$M, N, K = 5, 5, 2$}
\end{figure}
