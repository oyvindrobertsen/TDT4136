\section*{The Egg Carton Puzzle}
While trying to figure out how to implement the objective function, the observation was made that
\begin{quote}
    For an $M*N$ carton with limit $K$, the optimal solution will have \\
    $E = K * \min (M, N)$\\
    number of eggs.
\end{quote}
We used this to simplify the problem.
The start should be a carton with exactly $E$ eggs already distributed.
A neighboring carton is generated by swapping one egg to an open slot,
keeping the number of eggs constant.

With this known, our objective function does not need to penalize open slots in any way,
because we know the number of eggs is correct, it is just their placement that needs to be considered.

The objective function therefore penalizes one point per egg over $K$ in each row, column and diagonal.
The sum of all penalties is then compared to the absolute worst score achievable (eggs in every slot),
resulting in a score between 0 (worst) and 1 (optimal solution).

\newpage
\subsection*{Results}
Because of the non-deterministic nature of the algorithm,
it is not possible to recreate the same results every time.

Here are some sample results for the problems in the task,
showcasing different outcomes.

\subsubsection*{5, 5, 2}
\begin{figure}[h!]
\begin{multicols}{2}
    \begin{verbatim}
    . . O O . | 2
    . . O O . | 2
    . . O . O | 2
    . O . O O | 3
    O . . . . | 1
    - - - - -
    1 1 3 3 2
    Score: 0.875
    \end{verbatim}
    \columnbreak
    \begin{verbatim}
    . . O . O | 2
    O . O . . | 2
    O . . O . | 2
    . O . O . | 2
    . O . . O | 2
    - - - - -
    2 2 2 2 2
    Score: 1.0
    \end{verbatim}
    \end{multicols}
\caption{$M, N, K = 5, 5, 2$\\
Initial state left, last state right.}
\end{figure}
In this case an optimal solution was found.

\subsubsection*{6, 6, 2}
\begin{figure}[h!]
    \begin{verbatim}
    . . . . . . . . | 0
    . . . . . . O . | 1
    . . O . . . . . | 1
    . . . O . . . . | 1
    . . . . . . O . | 1
    . . . . . . O . | 1
    O . . . . . . . | 1
    . . O . . O . . | 2
    - - - - - - - -
    1 0 2 1 0 1 3 0
    Score: 0.9714285714285714

    O . . . . . . . | 1
    . . . . . O . . | 1
    . . . . . . . . | 0
    . O . . . . . . | 1
    . . . . . O O O | 3
    . . O . . . . . | 1
    . . O . . . . . | 1
    . . . . . . . . | 0
    - - - - - - - -
    1 1 2 0 0 2 1 1
    Score: 0.9809523809523809
    \end{verbatim}
\caption{$M, N, K = 8, 8, 1$}
\end{figure}


\newpage

\subsubsection*{8, 8, 1 (8 Queens Problem)}
\begin{figure}[h!]
    \begin{verbatim}
    O . . O . . | 2
    . . . O . O | 2
    O . . O . O | 3
    . . . . . . | 0
    . O O . O . | 3
    O . . . O . | 2
    - - - - - -
    3 1 1 3 2 2
    Score: 0.9375

    No solution with p.obj_func() > f_target was found, return last examined board.
    . . . O . O | 2
    . . O O O O | 4
    . . . O . . | 1
    . O . . O . | 2
    . . . . . O | 1
    . . . O . O | 2
    - - - - - -
    0 1 1 4 2 4
    Score: 0.8875
    \end{verbatim}
\caption{$M, N, K = 6, 6, 2$}
\end{figure}


\subsubsection*{10, 10, 3}
\begin{figure}[h!]
    \begin{verbatim}
    . . . O O . . . O O | 4
    . O . . . . . . . O | 2
    . . . . O O O . . . | 3
    . . . . . . . . . O | 1
    . O . O O . . . O . | 4
    . . . . . . . . O O | 2
    . O . . O . . . O O | 4
    O O . . . . . . . . | 2
    . O O . . . . O O . | 4
    . O O O O . . . . . | 4
    - - - - - - - - - -
    1 6 2 3 5 1 1 1 5 5
    Score: 0.9201680672268908

    No solution with p.obj_func() > f_target was found, return last examined board.
    O O . O . . . . O . | 4
    . O . . O . O . . O | 4
    . . . . . . . O . . | 1
    . . . . . O . . O . | 2
    O O . . O . . . . . | 3
    . . O . O O . . . O | 4
    O O O . O O . . . . | 5
    . . . . O . O . O . | 3
    . . . . O O . . . O | 3
    . . . . . . O . . . | 1
    - - - - - - - - - -
    3 4 2 1 6 4 3 1 3 3
    Score: 0.9411764705882353
    \end{verbatim}
\caption{$M, N, K = 8, 8, 1$}
\end{figure}

