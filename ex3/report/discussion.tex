\section{Problem A: Pathfinding in 2D Games}

All deliverables in the form of code are included aside the report.

\subsection*{Subproblem A.1: Grids with Obstacles}

The following figures show the solution path calculated by the A* implementation,
with the $\text{OPEN}$ and $\text{CLOSED}$ sets visualized as respectively blue
and red dots.

\begin{figure}[h!]
  \centering
    \includegraphics[width=0.5\textwidth]{img/board-1-1-astar}
    \caption{Board 1.1}
\end{figure}

\begin{figure}[h!]
  \centering
    \includegraphics[width=0.5\textwidth]{img/board-1-2-astar}
    \caption{Board 1.2}
\end{figure}

\begin{figure}[h!]
  \centering
    \includegraphics[width=0.5\textwidth]{img/board-1-3-astar}
    \caption{Board 1.3}
\end{figure}

\begin{figure}[h!]
  \centering
    \includegraphics[width=0.5\textwidth]{img/board-1-4-astar}
    \caption{Board 1.4}
\end{figure}

\newpage


\subsection*{Subproblem A.2: Grids with different cell costs}

For this subproblem, we modified our implementation to be able to parse weighted
boards as well as correctly handling cost calculation of weighted nodes.
Again, the visualizations include the $\text{OPEN}$ and $\text{CLOSED}$ sets.

\begin{figure}[h!]
  \centering
    \includegraphics[width=0.8\textwidth]{img/board-2-1-astar}
    \caption{Board 2.1}
\end{figure}

\begin{figure}[h!]
  \centering
    \includegraphics[width=0.8\textwidth]{img/board-2-2-astar}
    \caption{Board 2.2}
\end{figure}

\begin{figure}[h!]
  \centering
    \includegraphics[width=0.8\textwidth]{img/board-2-3-astar}
    \caption{Board 2.3}
\end{figure}

\begin{figure}[h!]
  \centering
    \includegraphics[width=0.8\textwidth]{img/board-2-4-astar}
    \caption{Board 2.4}
\end{figure}

\newpage

\subsection*{Comparison with BFS and Dijkstra's Algorithm}

\subsubsection*{Board 1.1}

\begin{figure}[h!]
  \centering
    \includegraphics[width=0.5\textwidth]{img/board-1-1-astar}
    \caption{Board 1.1 - A*}
\end{figure}

\begin{figure}[h!]
  \centering
    \includegraphics[width=0.5\textwidth]{img/board-1-1-dijkstra}
    \caption{Board 1.1 - Dijkstra}
\end{figure}

\begin{figure}[h!]
  \centering
    \includegraphics[width=0.5\textwidth]{img/board-1-1-bfs}
    \caption{Board 1.1 - BFS}
\end{figure}

All three algorithms find an equally short path, but A* examines significantly
fewer nodes doing it.

\newpage

\subsubsection*{Board 1.2}

\begin{figure}[h!]
  \centering
    \includegraphics[width=0.5\textwidth]{img/board-1-2-astar}
    \caption{Board 1.2 - A*}
\end{figure}

\begin{figure}[h!]
  \centering
    \includegraphics[width=0.5\textwidth]{img/board-1-2-dijkstra}
    \caption{Board 1.2 - Dijkstra}
\end{figure}

\begin{figure}[h!]
  \centering
    \includegraphics[width=0.5\textwidth]{img/board-1-2-bfs}
    \caption{Board 1.2 - BFS}
\end{figure}

All algorithms find the shortest path, but again, A* is much more efficient.

\newpage

\subsubsection*{Board 1.3}

\begin{figure}[h!]
  \centering
    \includegraphics[width=0.5\textwidth]{img/board-1-3-astar}
    \caption{Board 1.3 - A*}
\end{figure}

\begin{figure}[h!]
  \centering
    \includegraphics[width=0.5\textwidth]{img/board-1-3-dijkstra}
    \caption{Board 1.3 - Dijkstra}
\end{figure}

\begin{figure}[h!]
  \centering
    \includegraphics[width=0.5\textwidth]{img/board-1-3-bfs}
    \caption{Board 1.3 - BFS}
\end{figure}

Again, A* is much more efficient than the others.

\newpage

\subsubsection*{Board 1.4}

\begin{figure}[h!]
  \centering
    \includegraphics[width=0.5\textwidth]{img/board-1-4-astar}
    \caption{Board 1.4 - A*}
\end{figure}

\begin{figure}[h!]
  \centering
    \includegraphics[width=0.5\textwidth]{img/board-1-4-dijkstra}
    \caption{Board 1.4 - Dijkstra}
\end{figure}

\begin{figure}[h!]
  \centering
    \includegraphics[width=0.5\textwidth]{img/board-1-4-bfs}
    \caption{Board 1.4 - BFS}
\end{figure}

For this board, we see that there's not much difference between the three
algorithms. A* explores marginally fewer nodes than the other two.


