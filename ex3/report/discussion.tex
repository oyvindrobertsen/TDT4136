\section{Problem A: Pathfinding in 2D Games}

All deliverables in the form of code are enclosed.

\subsection*{Subproblem A.1: Grids with Obstacles}

The following figures show the solution path calculated by the A* implementation,
with the $\text{OPEN}$ and $\text{CLOSED}$ sets visualized as respectively cyan
and pink dots.

\begin{figure}[h!]
  \centering
    \includegraphics[width=0.8\textwidth]{img/board-1-1-astar}
    \caption{Board 1.1}
\end{figure}

\begin{figure}[h!]
  \centering
    \includegraphics[width=0.8\textwidth]{img/board-1-2-astar}
    \caption{Board 1.2}
\end{figure}

\begin{figure}[h!]
  \centering
    \includegraphics[width=0.8\textwidth]{img/board-1-3-astar}
    \caption{Board 1.3}
\end{figure}

\begin{figure}[h!]
  \centering
    \includegraphics[width=0.8\textwidth]{img/board-1-4-astar}
    \caption{Board 1.4}
\end{figure}

\clearpage

\subsection*{Subproblem A.2: Grids with different cell costs}

For this subproblem, we modified our implementation to be able to parse weighted
boards as well as correctly handling cost calculation of weighted nodes.
Again, the visualizations include the $\text{OPEN}$ and $\text{CLOSED}$ sets.

\begin{figure}[h!]
  \centering
    \includegraphics[width=\textwidth]{img/board-2-1-astar}
    \caption{Board 2.1}
\end{figure}

\begin{figure}[h!]
  \centering
    \includegraphics[width=\textwidth]{img/board-2-2-astar}
    \caption{Board 2.2}
\end{figure}

\begin{figure}[h!]
  \centering
    \includegraphics[width=\textwidth]{img/board-2-3-astar}
    \caption{Board 2.3}
\end{figure}

\begin{figure}[h!]
  \centering
    \includegraphics[width=\textwidth]{img/board-2-4-astar}
    \caption{Board 2.4}
\end{figure}

\clearpage

\subsection*{Comparison with BFS and Dijkstra's Algorithm}

\subsubsection*{Board 1.1}

\begin{figure}[h!]
  \centering
    \includegraphics[width=0.8\textwidth]{img/board-1-1-astar}
    \caption{Board 1.1 - A*}
\end{figure}

\begin{figure}[h!]
  \centering
    \includegraphics[width=0.8\textwidth]{img/board-1-1-dijkstra}
    \caption{Board 1.1 - Dijkstra}
\end{figure}

\begin{figure}[h!]
  \centering
    \includegraphics[width=0.8\textwidth]{img/board-1-1-bfs}
    \caption{Board 1.1 - BFS}
\end{figure}

All three algorithms find an equally short path, but A* examines significantly
fewer nodes doing it.

\clearpage

\subsubsection*{Board 1.2}

\begin{figure}[h!]
  \centering
    \includegraphics[width=0.8\textwidth]{img/board-1-2-astar}
    \caption{Board 1.2 - A*}
\end{figure}

\begin{figure}[h!]
  \centering
    \includegraphics[width=0.8\textwidth]{img/board-1-2-dijkstra}
    \caption{Board 1.2 - Dijkstra}
\end{figure}

\begin{figure}[h!]
  \centering
    \includegraphics[width=0.8\textwidth]{img/board-1-2-bfs}
    \caption{Board 1.2 - BFS}
\end{figure}

All algorithms find the shortest path, but again, A* is much more efficient.

\clearpage

\subsubsection*{Board 1.3}

\begin{figure}[h!]
  \centering
    \includegraphics[width=0.8\textwidth]{img/board-1-3-astar}
    \caption{Board 1.3 - A*}

\end{figure}

\begin{figure}[h!]
  \centering
    \includegraphics[width=0.8\textwidth]{img/board-1-3-dijkstra}
    \caption{Board 1.3 - Dijkstra}
\end{figure}

\begin{figure}[h!]
  \centering
    \includegraphics[width=0.8\textwidth]{img/board-1-3-bfs}
    \caption{Board 1.3 - BFS}
\end{figure}

Again, A* is much more efficient than the others.

\clearpage

\subsubsection*{Board 1.4}

\begin{figure}[h!]
  \centering
    \includegraphics[width=0.8\textwidth]{img/board-1-4-astar}
    \caption{Board 1.4 - A*}
\end{figure}

\begin{figure}[h!]
  \centering
    \includegraphics[width=0.8\textwidth]{img/board-1-4-dijkstra}
    \caption{Board 1.4 - Dijkstra}
\end{figure}

\begin{figure}[h!]
  \centering
    \includegraphics[width=0.8\textwidth]{img/board-1-4-bfs}
    \caption{Board 1.4 - BFS}
\end{figure}

For this board, we see that there's not much difference between the three
algorithms. A* explores marginally fewer nodes than the other two.

\clearpage

\subsubsection*{Board 2.1}

\begin{figure}[h!]
  \centering
    \includegraphics[width=\textwidth]{img/board-2-1-astar}
    \caption{Board 2.1 - A*}
\end{figure}

\begin{figure}[h!]
  \centering
    \includegraphics[width=\textwidth]{img/board-2-1-dijkstra}
    \caption{Board 2.1 - Dijkstra}
\end{figure}

\begin{figure}[h!]
  \centering
    \includegraphics[width=\textwidth]{img/board-2-1-bfs}
    \caption{Board 2.1 - BFS}
\end{figure}

For this board, we see that A* and Djikstra both find equally optimal paths,
both processing about the same number of nodes. (A* slightly lower as the 
heuristic is taken into consideration.) BFS processes fewer nodes, but finds an
extremely unoptimized path.

\clearpage

\subsubsection*{Board 2.2}

\begin{figure}[h!]
  \centering
    \includegraphics[width=\textwidth]{img/board-2-2-astar}
    \caption{Board 2.2 - A*}
\end{figure}

\begin{figure}[h!]
  \centering
    \includegraphics[width=\textwidth]{img/board-2-2-dijkstra}
    \caption{Board 2.2 - Dijkstra}
\end{figure}

\begin{figure}[h!]
  \centering
    \includegraphics[width=\textwidth]{img/board-2-2-bfs}
    \caption{Board 2.2 - BFS}
\end{figure}

A* and Dijkstra both perform reasonably well here, while BFS makes quite a few
suboptimal choices while also processing more nodes.

For the case of A* vs. Djikstra, we see that A* (as it should) refrains from
processing nodes that have low cell weight, but also have higher heuristic values.

\clearpage

\subsubsection*{Board 2.3}

\begin{figure}[h!]
  \centering
    \includegraphics[width=\textwidth]{img/board-2-3-astar}
    \caption{Board 2.3 - A*}
\end{figure}

\begin{figure}[h!]
  \centering
    \includegraphics[width=\textwidth]{img/board-2-3-dijkstra}
    \caption{Board 2.3 - Dijkstra}
\end{figure}

\begin{figure}[h!]
  \centering
    \includegraphics[width=\textwidth]{img/board-2-3-bfs}
    \caption{Board 2.3 - BFS}
\end{figure}

Again, A* and Dijkstra both find equally optimal paths, while BFS naively picks
the first path it can find.

\clearpage

\subsubsection*{Board 2.4}

\begin{figure}[h!]
  \centering
    \includegraphics[width=\textwidth]{img/board-2-4-astar}
    \caption{Board 2.4 - A*}
\end{figure}

\begin{figure}[h!]
  \centering
    \includegraphics[width=\textwidth]{img/board-2-4-dijkstra}
    \caption{Board 2.4 - Dijkstra}
\end{figure}

\begin{figure}[h!]
  \centering
    \includegraphics[width=\textwidth]{img/board-2-4-bfs}
    \caption{Board 2.4 - BFS}
\end{figure}

For this final board, A* and Dijkstra find the same path. A* processes slightly
fewer nodes, notably those in the bottom right corner of the board.
BFS processes few nodes, but finds a suboptimal path.
